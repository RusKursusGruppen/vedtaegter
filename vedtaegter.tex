% vim: spell spelllang=da_dk

\documentclass[a4paper, 10pt]{article}

% Tekstindkodning og orddeling
\usepackage[utf8]{inputenc}
\usepackage[danish, english]{babel}

\title{Vedtægter for foreningen RKG - Ruskursusgruppen}

\def\vedtagsdato{den 18. november 2016}

\author{}
\date{Som vedtaget \vedtagsdato{}}

% Palatino skrifttype
\usepackage[T1]{fontenc}
\usepackage{mathpazo}
\linespread{1.05}

\renewcommand\thesection{\textsection\arabic{section}}

% Fint hoved og fod
\usepackage{fancyhdr}
\usepackage{lastpage}
\renewcommand{\headrulewidth}{0in}
\renewcommand{\headsep}{40pt}
\setlength{\headheight}{25pt}
\pagestyle{fancy}
\makeatletter
\let\headtitle\title
\let\headdate\date
\makeatother
\lhead{\headtitle}
\rhead{\headdate\\}
\cfoot{Side~\thepage~af~\pageref{LastPage}}
\fancypagestyle{first}{%
  \fancyhf{}%
  \cfoot{Side~\thepage~af~\pageref{LastPage}}%
}

\usepackage{enumitem}

\newenvironment{stykenum}{
  \begin{enumerate}[%
    label=Stk.~\arabic*:, ref=\textsection~\theenumi~Stk.~\arabic*, start=1]
}{\end{enumerate}}

\begin{document}

\maketitle
\thispagestyle{first}

\section{Navn og hjemsted}
\begin{stykenum}
    \item \label{navn} Foreningens navn er Foreningen af rusvejledere på
        Datalogisk Institut af 2005, I daglig tale Ruskursusgruppen.

    \item \label{hjemsted} Foreningen har hjemsted Universitetsparken 1,
        Datalogisk Institut ved Københavns Universitet.
\end{stykenum}

\section{Formål}
\begin{stykenum}
\item \label{formaal} Foreningens formål er at planlægge og udføre de
    studerendes bidrag til studieintroduktionen for nye studerende på Datalogisk
    Institut ved Københavns Universitet, samt at forbedre studiemiljøet på tværs
    af alle årgange.
\end{stykenum}

\section{Medlemsforhold}
\begin{stykenum}
    \item Som medlem kan enhver studerende på Datalogisk Institut ved Københavns
        Universitet optages.

    \item Indmeldelse sker ved henvendelse til formand eller et andet
        bestyrelsesmedlem. Ved indmeldelse betales forholdsmæssigt kontingent
        for den resterende del af året/fuldt kontingent for det resterende år.

    \item Medlemmerne er forpligtet til at overholde foreningens love og
        vedtægter.

    \item Medlemmerne er berettiget til at møde på generalforsamlingen og stemme
        om forslag, der rejses på generalforsamlingen.

    \item Medlemmerne er berettiget til/har ret til at indkalde til
        ekstraordinær generalforsamling, såfremt mindst halvdelen af medlemmerne
        giver udtryk for ønske om dette.

    \item Medlemskontingent fastsættes for et år ad gangen af
        generalforsamlingen.

    \item Udmeldelse af foreningen sker skriftligt til enten formanden eller
        kasseren.

    \item Generalforsamlingen kan beslutte at ekskludere et medlem, som ikke
        opfylder betingelserne for medlemskabet, eller som handler til skade for
        foreningen. Beslutning om ekskludering af et medlem af foreningen
        træffes på generalforsamlingen med 2/3 flertal efter samme procedure som
        vedtægtsændringer.

    \item Uanset ovenstående vil et medlem automatisk blive ekskluderet fra
        foreningen, hvis medlemmet ikke betaler kontingent.

    \item Generalforsamlingen kan vælge at optage æresmedlemmer. Et
        æresmedlemskab giver automatisk opstillings-, tale og stemmeret
        ummidelbart efter optagelse og frem til næste ordinære
        generalforsamling.
\end{stykenum}

\section{Generalforsamlingen}
\begin{stykenum}
    \item Generalforsamlingen er foreningens øverste myndighed.

    \item Ordinær generalforsamling afholdes én gang årligt i 4. kvartal.

    \item Mødeberettiget på generalforsamlingen er alle medlemmer, der har
        betalt kontingent for pågældende år.

    \item Stemmeret på generalforsamlingen har foreningens medlemmer, og
        medlemmernes stemmer vægtes lige.

    \item Beslutninger på generalforsamlingen træffes med almindeligt flertal,
        medmindre andet følger af denne vedtægt.
\end{stykenum}

\section{Indkaldelse til generalforsamling}
\begin{stykenum}
    \item Indkaldelse til ordinær generalforsamling skal ske med mindst 2 ugers
        varsel. Med indkaldelsen skal følge en dagsorden for generalforsamlingen
        jf. stk. 8. Indkaldelsen offentliggøres via foreningens e-postliste.

    \item Forslag, der ønskes behandlet på generalforsamlingen, skal være
        formanden i hænde senest 1 uge før generalforsamlingens afholdelse.

    \item Dagsorden for ordinær generalforsamling skal indeholde mindst følgende
        punkter:
        \begin{enumerate}[label=\arabic*.]
            \item Valg af ordstyrer
            \item Valg af to stemmetællere
            \item Formandens beretning
            \item Kasserens beretning
            \item Forelæggelse og godkendelse af årsregnskab
            \item Fastsættelse af kontingent
            \item Uddeling af æresmedlemskaber jf.~stk.~§ 3, stk. 10
            \item Indkomne forslag
            \item Valg af formand og næstformand
            \item Valg af kasser, samt evt.\ assistenter
            \item Evt.
        \end{enumerate}

    \item Generalforsamlingen vælger ordstyrer til at lede forhandlingerne.
        Generalforsamlingens beslutninger indføres til protokol.
\end{stykenum}

\section{Ekstraordinær generalforsamling}
\begin{stykenum}
    \item Ekstraordinær generalforsamling kan til enhver tid indkaldes af
        bestyrelsen, eller hvis mindst halvdelen af foreningens medlemmer ønsker
        det. Indkaldelse skal ske med mindst 2 ugers varsel og skal indeholde en
        dagsorden for generalforsamlingen.
\end{stykenum}

\section{Foreningens daglige ledelse}
\begin{stykenum}
    \item Foreningens daglige ledelse varetages af bestyrelsen på mindst 3 og
        højest 5 medlemmer, valgt på generalforsamlingen for 1 år ad gangen.

    \item Bestyrelsen leder foreningen i overensstemmelse med nærværende
        vedtægter og generalforsamlingens beslutninger.

    \item Formand, næstformand, kasser og nul til to bestyrelsesmedlemmer vælges
        hvert år.

    \item Bestyrelsen fastsætter selv sin forretningsorden.

    \item Formanden leder mødet og i dennes frafald næstformanden.

    \item Bestyrelsen er beslutningsdygtig, når mindst halvdelen af
        bestyrelsesmedlemmerne er til stede.

    \item Bestyrelsen træffer beslutninger ved simpelt flertal. Ved stemmelighed
        er formanden eller den fungerende formands stemme afgørende.

    \item Der føres beslutningsreferat over bestyrelsesmøderne, som underskrives
        af formanden, næstformanden og referenten.
\end{stykenum}

\section{Tegningsregler}
\begin{stykenum}
    \item Foreningen tegnes af formand og kasser.
\end{stykenum}

\section{Hæftelse}
\begin{stykenum}
    \item Foreningen hæfter for sine forpligtelser med den af foreningen til
        enhver tid tilhørende formue. Der påhviler ikke foreningens medlemmer
        eller bestyrelsen nogen personlig hæftelse, andet end hvis dette er
        særlig aftalt.
\end{stykenum}

\section{Regnskab og revision}
\begin{stykenum}
    \item Foreningens regnskabs- og kontingentår følger kalenderåret.

    \item Foreningens formue skal anbringes i anerkendt pengeinstitut.

    \item Kassereren inkasserer kontingent og betaler de af bestyrelsen godkendte
    regninger. Han fører regnskab over samtlige indtægter og udgifter i en sådan
    form, at foreningens øjeblikkelige økonomi altid kan aflæses. Kassereren
    udarbejder foreningens årsregnskab.
\end{stykenum}

\section{Vedtægtsændringer}
\begin{stykenum}
    \item Ændring af nærværende vedtægt kræver vedtagelse på generalforsamling
        med 2/3 af de afgivne stemmer.
\end{stykenum}

\section{Opløsning}
\begin{stykenum}
    \item Foreningen kan kun opløses på en generalforsamling, som er indkaldt
        med dette for øje.

    \item Opløsning af foreningen kræver vedtagelse med 2/3 majoritet af
        samtlige foreningens medlemmer. Opnås denne majoritet ikke, er
        bestyrelsen berettiget til at indkalde til en ny generalforsamling, på
        hvilken opløsning kan vedtages med 2/3 majoritet af de fremmødte
        medlemmer.

    \item Eventuelt overskud/formue bliver ved opløsningen af foreningen delt
        ligeligt imellem alle medlemmer.
\end{stykenum}
\end{document}
