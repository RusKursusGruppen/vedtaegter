% vim: spell spelllang=da_dk

\documentclass[a4paper, 10pt]{article}

% Tekstindkodning og orddeling
\usepackage[utf8]{inputenc}
\usepackage[danish, english]{babel}

\title{Vedtægter for foreningen RKG - Ruskursusgruppen}

\def\vedtagsdato{den 18. november 2016}

\author{}
\date{Som vedtaget \vedtagsdato{}}

% Palatino skrifttype
\usepackage[T1]{fontenc}
\usepackage{mathpazo}
\linespread{1.05}

\renewcommand\thesection{\textsection\arabic{section}}

% Fint hoved og fod
\usepackage{fancyhdr}
\usepackage{lastpage}
\renewcommand{\headrulewidth}{0in}
\renewcommand{\headsep}{40pt}
\setlength{\headheight}{25pt}
\pagestyle{fancy}
\makeatletter
\let\headtitle\title
\let\headdate\date
\makeatother
\lhead{\headtitle}
\rhead{\headdate\\}
\cfoot{Side~\thepage~af~\pageref{LastPage}}
\fancypagestyle{first}{%
  \fancyhf{}%
  \cfoot{Side~\thepage~af~\pageref{LastPage}}%
}

\usepackage{enumitem}

\newenvironment{stykenum}{
  \begin{enumerate}[%
    label=Stk.~\arabic*:, ref=\thesection~Stk.~\arabic{enumi}, start=1]
}{\end{enumerate}}

\newenvironment{substykenum}{
  \begin{enumerate}[%
          label=Stk.~\arabic{enumi}.\arabic*:,
      ref=\thesection~Stk.~\arabic{enumi}.\arabic*, start=1]
}{\end{enumerate}}

\newenvironment{subsubstykenum}{
  \begin{enumerate}[%
      label=Stk.~\arabic{enumi}.\arabic{enumii}.\arabic*:,
      ref=\thesection~Stk.~\arabic{enumi}.\arabic{enumii}.\arabic*, start=1]
}{\end{enumerate}}
\begin{document}

\maketitle
\thispagestyle{first}

\section{Navn og hjemsted}
\begin{stykenum}
    \item \label{navn} Foreningens navn er RusKursusGruppen. I daglig tale RKG.

    \item \label{hjemsted} Foreningens hjemsted er Universitetsparken 1,
        Datalogisk Institut ved Københavns Universitet.

\end{stykenum}

\section{Formål}
\begin{stykenum}
\item \label{formaal} Foreningens formål er at planlægge og afholde  rustur, at
    være til stede under campusdagene, at afholde sociale arrangementer for
    førsteårsstuderende i løbet af studieåret,  samt at forbedre studiemiljøet
    på tværs af alle årgange.
\end{stykenum}

\section{Medlemsforhold}
\begin{stykenum}
    \item Som medlem kan enhver studerende på Datalogisk Institut ved Københavns
        Universitet optages.

    \item Indmeldelse sker ved henvendelse til formand eller et andet
        bestyrelsesmedlem. Ved indmeldelse betales fuldt kontingent for det
        resterende år.

    \item Medlemmerne er forpligtet til at overholde foreningens love og
        vedtægter.

    \item Aktive medlemmer er berettiget til at møde på generalforsamlingen og
        stemme om forslag, der rejses på generalforsamlingen.
        \begin{substykenum}
            \item Aktive medlemmer har enten deltaget i afholdelse af rustur
                og/eller campusdage det pågældende år som vejleder, eller er
                æresmedlemmer.
        \end{substykenum}

    \item Aktive medlemmer  er berettiget til/har ret til at indkalde til
        ekstraordinær generalforsamling, såfremt mindst halvdelen af de aktive
        medlemmer giver udtryk for ønske om dette.

    \item Medlemskontingent fastsættes for et år ad gangen af
        generalforsamlingen.

    \item Udmeldelse af foreningen sker skriftligt til enten formanden eller
        kasseren.

    \item Generalforsamlingen kan beslutte at ekskludere et medlem, som ikke
        opfylder betingelserne for medlemskabet, eller som handler til skade for
        foreningen. Beslutning om ekskludering af et medlem af foreningen
        træffes på generalforsamlingen med 2/3 flertal efter samme procedure som
        vedtægtsændringer.
        \begin{substykenum}
        \item Et flertal i bestyrelsen kan ekskludere et medlem i 2 uger, indtil
            næste  generalforsamling.
        \end{substykenum}

    \item Uanset ovenstående vil et medlem automatisk blive ekskluderet fra
        foreningen, hvis medlemmet ikke betaler kontingent.

    \item Et medlem kan ekskluderes øjeblikkeligt såfremt medlemmet indleder
        seksuel aktivitet med en ny studerende inden afholdelse af
        førsteårsfesten.
        \begin{substykenum}
            \item Afholdes førsteårsfesten ikke, er gældende tidsfrist 1.
                November kl. 13:37.

            \item Mistænkes et medlem for at have brudt stk. 10, indkaldes
                medlemmet til et møde med bestyrelsen, hvor den endelige
                beslutning tages af bestyrelsen.

            \item Hvis det er første gang et medlem ekskluderes af ovenstående
                årsag, har medlemmet karantæne fra RKG i et år. Hvis det ikke er
                første gang, er medlemmet ude af RKG for evigt.
        \end{substykenum}

    \item Foreningens Facebook-gruppe skal reflektere foreningens medlemmer.
        Dette inkluderer aktive medlemmer, hyttebumser og eventuelle
        førsteårsvejledere.
        \begin{substykenum}
            \item Førsteårsvejledere kan tilføjes efter dværgeaften.

            \item Facebook-gruppens medlemmer opdateres løbende. Som minimum
                sker det efter 2. Nar.

            \item Bestyrelsen kan til enhver tid ignorere ovenstående
                regler.

        \end{substykenum}

    \item \label{aeresmedlemmer} Generalforsamlingen kan vælge at optage
      æresmedlemmer. Et æresmedlemskab giver automatisk opstillings-, tale og
      stemmeret ummidelbart efter optagelse og frem til næste ordinære
      generalforsamling.
\end{stykenum}

\section{Generalforsamlingen}
\begin{stykenum}
    \item Generalforsamlingen er foreningens øverste myndighed.

    \item Ordinær generalforsamling afholdes én gang årligt i 3. kvartal.

    \item Mødeberettiget på generalforsamlingen er alle medlemmer, der har
        betalt kontingent for pågældende år.

    \item Stemmeret på generalforsamlingen har foreningens aktive medlemmer, og
        medlemmernes stemmer vægtes lige.

    \item Beslutninger på generalforsamlingen træffes med almindeligt flertal,
        medmindre andet følger af denne vedtægt.
\end{stykenum}

\section{Indkaldelse til generalforsamling}
\begin{stykenum}
    \item Indkaldelse til ordinær generalforsamling skal ske med mindst 2 ugers
        varsel. Med indkaldelsen skal følge en dagsorden for generalforsamlingen
        jf. \ref{dagsorden}. Indkaldelsen offentliggøres via foreningens
        Facebook-gruppe.

    \item Forslag, der ønskes behandlet på generalforsamlingen, skal være
        formanden i hænde senest 1 uge før generalforsamlingens afholdelse.

    \item \label{dagsorden} Dagsorden for ordinær generalforsamling skal
        indeholde mindst følgende punkter:
        \begin{enumerate}[label=\arabic*.]
            \item Valg af ordstyrer og referent.

            \item Valg af to stemmetællere.

            \item Formandens beretning.

            \item Kasserens beretning.

            \item Forelæggelse og godkendelse af årsregnskab.

            \item Fastsættelse af kontingent.

            \item Uddeling af æresmedlemskaber jf.~\ref{aeresmedlemmer}

            \item Indkomne forslag

            \item Valg af formand og næstformand

            \item 10. Valg af kasser, samt evt.  assistenter.

            \item Valg af antal af menige medlemmer i bestyrelsen samt deres
                roller.

            \item Valg af menige medlemmer.

            \item Evt.
        \end{enumerate}
\end{stykenum}

\section{Ekstraordinær generalforsamling}
\begin{stykenum}
    \item Ekstraordinær generalforsamling kan til enhver tid indkaldes af
        bestyrelsen, eller hvis mindst halvdelen af foreningens medlemmer ønsker
        det. Indkaldelse skal ske med mindst 2 ugers varsel og skal indeholde en
        dagsorden for generalforsamlingen.
\end{stykenum}

\section{Foreningens daglige ledelse}
\begin{stykenum}
    \item Foreningens daglige ledelse varetages af bestyrelsen på mindst 3
        medlemmer, valgt på generalforsamlingen for 1 år ad gangen.

    \item Bestyrelsen leder foreningen i overensstemmelse med nuværende
        vedtægter og generalforsamlingens beslutninger.

    \item Formand, næstformand, kasser og mindst 0 menige bestyrelsesmedlemmer
        vælges hvert år.

    \item Bestyrelsen fastsætter selv sin forretningsorden.

    \item Formanden leder mødet og i dennes frafald næstformanden.

    \item Bestyrelsen er beslutningsdygtig, når mindst halvdelen af
        bestyrelsesmedlemmerne er til stede.

    \item Bestyrelsen træffer beslutninger ved simpelt flertal. Ved stemmelighed
        er formanden eller den fungerende formands stemme afgørende.

    \item Der føres beslutningsreferat over bestyrelsesmøderne, som underskrives
        af formanden, næstformanden og referenten.
\end{stykenum}

\section{Deltagelse i rustur}
\begin{stykenum}
\item Deltagelse i rustur som vejleder sker efter følgende kriterier
    \begin{substykenum}
        \item Medlemmer skal deltage i 2. Nar
        \begin{subsubstykenum}
            \item Er medlemmet forhindret i deltagelse af 2. Nar, skal der
                opfyldes andet optagelseskrav, der er opstillet af bestyrelsen.
        \end{subsubstykenum}

        \item Medlemmet skal have deltaget i mindst 1 internt arrangement,
            afholdt af Ruskursusgruppen.
    \end{substykenum}

    \item Deltagelse i rustur som hyttebums sker efter følgende kriterier:
    \begin{substykenum}
        \item Som udgangspunkt skal alle hyttebumser have tilmeldt sig inden
            2.nar.
        \item Hyttebumser kan stille op som hold, og kan som udgangspunkt ikke
            splittes op.
        \item I ekstraordinære tilfælde, kan man blive hyttebums ved tilmeldelse
            efter 2. Nar.
    \end{substykenum}
    \item Hvert vejlederhold afgiver en prioriteret liste over hyttebumser.
        Bestyrelsen fungerer som puslegruppe.
        \begin{substykenum}
            \item Hvis et bestyrelsesmedlem vil være hyttebums, er denne ikke
                med til at pusle.
        \end{substykenum}
    \item Afstemninger om hold af hyttebumser skal ske ved anonym afstemning,
        hvor der kræves enstemmig ja, for at hyttebumserne kan godkendes.
        \begin{substykenum}
            \item Tilmelding som hyttebums efter fristen kræver en anonym
                afstemning blandt det givne vejlederhold, samt allerede
                tilknyttede hyttebumser. Resultatet skal være et enstemmig ja.
        \end{substykenum}
\end{stykenum}

\section{Pusling}
\begin{stykenum}
    \item Pusleholdet skal bestå af tre personer, offentligt valgt på 2. Nar.
        \begin{substykenum}
            \item Mindst én pusler skal have puslet tidligere. Hvis ikke dette
                er muligt, skal en tidligere pusler introducere puslingen til
                det nye puslehold, men uden at få noget fortroligt information.
        \end{substykenum}
    \item Puslingen er strengt fortroligt, og eventuelle brud på denne
        tavshedspligt behandles af bestyrelsen.
\end{stykenum}

\section{Tegningsregler}
\begin{stykenum}
    \item Foreningen tegnes af formand og kasser.
\end{stykenum}

\section{Hæftelse}
\begin{stykenum}
    \item Foreningen hæfter for sine forpligtelser med den af foreningen til
        enhver tid tilhørende formue. Der påhviler ikke foreningens medlemmer
        eller bestyrelsen nogen personlig hæftelse, andet end hvis dette er
        særlig aftalt.
\end{stykenum}

\section{Regnskab og revision}
\begin{stykenum}
    \item Foreningens regnskabs- og kontingentår følger kalenderåret.

    \item Foreningens formue skal anbringes i anerkendt pengeinstitut.

    \item Kassereren indkasserer kontingent og betaler de af bestyrelsen
        godkendte regninger. Vedkommende fører regnskab over samtlige indtægter
        og udgifter i en sådan form, at foreningens øjeblikkelige økonomi altid
        kan aflæses. Kassereren udarbejder foreningens årsregnskab.
\end{stykenum}

\section{Vedtægtsændringer}
\begin{stykenum}
    \item Ændring af nuværende vedtægt kræver vedtagelse på generalforsamling
        med 2/3 af de afgivne stemmer.
    \begin{substykenum}
        \item Blanke stemmer tælles ikke med i antallet af afgivne stemmer.
    \end{substykenum}
\end{stykenum}

\section{Opløsning}
\begin{stykenum}
    \item Foreningen kan kun opløses på en generalforsamling, som er indkaldt
        med dette for øje.

    \item Opløsning af foreningen kræver vedtagelse med 2/3 majoritet af
        samtlige foreningens aktive medlemmer. Opnås denne majoritet ikke, er
        bestyrelsen berettiget til at indkalde til en ny generalforsamling, på
        hvilken opløsning kan vedtages med 2/3 majoritet af de fremmødte
        medlemmer.

    \item Eventuelt overskud/formue bliver ved opløsningen af foreningen delt
        ligeligt imellem alle medlemmer.
\end{stykenum}
\end{document}
